\documentclass[11pt]{article}

\title{How to annotate a bibliography}

\author{Richard Lange}

\begin{document}

%%%%%%%%%%%%%
% Main text %
%%%%%%%%%%%%%

\maketitle

In the text, cite things as normal\cite{latexcompanion}. It works for single\cite{einstein} or for more\cite{knuthwebsite,latexcompanion} citations in a group.

%%%%%%%%%%%%%%%%%%%%%%%%%%%%%%
% Bibliography configuration %
%%%%%%%%%%%%%%%%%%%%%%%%%%%%%%

% Set the title of the bibliography page
\renewcommand{\refname}{References and recommended reading}

% Link to the .bib file containing annotations
\bibliography{sources_and_annotations}

% Link to modified style file.
\bibliographystyle{currbiol}

% All further config is handled by currbiol.bst, which looks for the 'outstanding' or 'special' fields in a bib entry

\null

\end{document}